\documentclass[a4paper,12pt, openany]{book}
\usepackage[utf8]{inputenc}
\usepackage{graphicx}

\usepackage[spanish, es-tabla]{babel}
\usepackage{color, colortbl}
\definecolor{Gray}{gray}{0.9}
\usepackage{amsfonts}
\usepackage{verbatim} % comentarios
\usepackage{fancyhdr} % Encabezados
\usepackage{tabularx}
\usepackage{hyperref}
\usepackage{ amssymb }
\usepackage{multirow}
\usepackage{pdflscape}
\usepackage{makecell}
\usepackage{amsmath}
\usepackage{subfig}

\renewcommand{\arraystretch}{1.5}

\renewcommand{\chaptername}{Capítulo}
\renewcommand{\contentsname}{Contenido}

\def\TITULO{Redes generativas adversarias para la segmentación de nervios periféricos} 
\def\FECHA{Fecha} 


\begin{document}

{
\thispagestyle{empty}

\begin{center}
\begin{table}[!htb]
\begin{tabular}{p{2.4cm}p{15cm}}
\includegraphics[width=2.4cm]{images/logoUQ.png}
\begin{center}
\vspace{0.5cm}
\rule[2cm]{0.8mm}{ 0.95\textheight}%vertical
\hspace{1pt}
\rule[2cm]{0.4mm}{0.95\textheight}%vertical
\end{center}
&
\vspace{-3.5cm}
\begin{center}
\rule[1mm]{0.95\linewidth}{0.4mm}%horizontal
\vspace{0.1pt}
\rule[3mm]{0.95\linewidth}{0.8mm}%horizontal
\\
\vspace*{0.3cm}
\LARGE{\bf{[Universidad]}}\\
\vspace*{0.3cm}
\large{\bf{[Facultad}}

\vspace{3\baselineskip}
{\Large \bf{[Titulo]}}\\
\vspace*{0.2cm}
\vspace*{1.6cm}
\Large{\bf [Trabajo de grado para optar al título de] }
\Large{\bf [Grado] }


\vspace*{1.5cm}
{\large \bf{Presentado por:}}

\vspace*{0.5cm}
{ \bf{[Autor 1]}}\\
{ \bf{[Autor 2]}}

\vspace*{1.5cm}
{\large \bf{Director:}}

\vspace*{0.5cm}
{ \bf{[Director], [PhD, Msc]}}\\



\vspace*{3cm}

{\large \bf{[Ciudad], [Departamento], [País]}}\\
\vspace*{-0.1cm}
{\large \bf{[Año]}}
\end{center}
\end{tabular}
\end{table}
\end{center}
}
\newenvironment{acknowledgements}%
    {\thispagestyle{empty}\null\vfill\begin{center}%
    \bfseries AGRADECIMIENTOS\end{center}}%
    {~\vfill}
        \begin{acknowledgements}
   		\textit{``[Agradecimientos autor 1]''.}
   		\begin{flushright}
   			\textbf{[Autor 1]}
   		\end{flushright}
   		\textit{``[Agradecimientos autor 2]''}
   		\begin{flushright}
   			\textbf{[Autor 2]}
   		\end{flushright}
   		
   		
   		
        \end{acknowledgements}
\newenvironment{resumen}%
    {\thispagestyle{empty}\null\vfill\begin{center}%
    \bfseries RESUMEN\end{center}}%
    {~\vfill}
        \begin{resumen}
   		[Resumen]
   		
        \end{resumen}
\newenvironment{abstract}%
    {\thispagestyle{empty}\null\vfill\begin{center}%
    \bfseries ABSTRACT\end{center}}%
    {~\vfill}
        \begin{abstract}
   		[Abstract]
   		
        \end{abstract}
{\thispagestyle{empty}\begin{center}%
		\bfseries GLOSARIO\end{center}}%
\begin{itemize}
	\item NN: Neural Network
	\item GAN: Generative Adversarial Network.
\end{itemize}


\tableofcontents

\chapter{PLANTEAMIENTO DEL PROBLEMA}
		\section{[Problemática a resolver]}		
		\section{Formulación del problema de investigación}
		
		
\section{JUSTIFICACIÓN}
		% Pertinencia: porque es pertinente
		\subsection{Pertinencia}
		
		% Viabilidad: (que grupos han trabajado en tematicas similares, porque es viable?
		\subsection{Viabilidad}
		
		% Impacto: el que tendra tanto el deasrrollo metodológico, como la aplicaci ́on en la salud
		\subsection{Impacto}

\include{pages/estado_del_arte/estado_arte}
	\nopagebreak[0]
	\clearpage
	\chapter{OBJETIVOS}
		\section{General}
			[Objetivo General]
	
	\section{Específicos}
		\begin{enumerate}
			\item [Objetivo especifico 1]
			\item [Objetivo especifico 2]
			\item [Objetivo especifico 3]
		\end{enumerate}
\chapter{MATERIALES Y MÉTODOS}

	\section{Bases de Datos}
	
	\section{Caja de Herramientas}

\chapter{RESULTADOS Y DISCUSIÓN} \label{sec:resultados}
	
\chapter{CONCLUSIONES Y TRABAJOS FUTUROS}
	
\chapter{PUBLICACIONES}
El sistema automático para la segmentación de nervios periféricos se presentó en el segundo simposio de investigación Comfamiliar 2019 bajo la modalidad de ponencia el día 24 de octubre de 2019, llevado a cabo en la ciudad de Pereira, Risaralda, donde se presentaron diferentes estudios de investigación desarrollados en el área de la medicina frente a una comunidad científica multidisciplinar.

\begin{itemize}
	\item Jiménez G. Andrés, Rodriguez M. Wilson, García A. Hernán \textit{Redes Generativas Adversarias para la Segmentación de Nervios Periféricos.} SIC 2019. Pereira-Risaralda.
\end{itemize}



% Hacer capitulo de publicaciones (simposio)

\backmatter
% bibliography, glossary and index would go here.


\bibliographystyle{ieeetr}
\bibliography{bib/biblio}

\end{document}